\section{Analysis} \label{sec:analysis} 

As it is shown in section~\ref{sec:evaluation}, Adjacency Filtering reduces the
number of edit-distance performed a lot.  However, as the Adjacency Filtering
will always test if N – e segments find corresponding locations, the
effectiveness of Adjacency Filtering is related to the user-set error tolerance
number e. As e increases, the effectiveness of Adjacency Filtering decreases,
since the required number of matching segments is reduced. To maintain the
Adjacency Filtering effectiveness, one solution will be using shorter key for
hash table thus cut the fragments into shorter segments and increasing the
segments count.  For example, if the fragment is 108 base-pairs in length and
each segment is 12 base-pair in length so we cut the fragment into 9 segments,
for a user set error tolerance number e=5, instead of requiring N-e = 9-5 = 4
segments find corresponding adjacent locations, we now reduce the key length to
9 which will divide the fragment into 12 segments, and we will require N-e =
12-5 = 7 segments finding corresponding adjacent locations. We have simulated 9
case for 1 chromosome and it turns out when e=5, setting key length to 9 does
increase the effectiveness of Adjacency Filtering where it is filtering out
more not matching locations. However, we do see a longer execution time. The
reason is that although the Adjacency Filtering effectiveness is enhanced, the
cost of Adjacency Filtering is increased. Since now the keys are shorter in
hash table and the entries are fewer while the total number of coordinates
stays the same as before, the coordinate list for each hash table entry is
increased. As a result, Adjacency Filtering will need to reverse more
coordinates. To make it even worse, now there are more segments and thus more
searching for corresponding adjacent coordinates. Last but not the least, this
increase the chance of encountering “popular” keys with the same reason we
described in Algorithm section, as now we are having less entries but longer
coordinate list. In all, this reduces the execution time a lot. In Figure 7, we
show when changing key length from 12 to 9. How many coordinate are subjects to
Adjacency Filtering compared to before. In Figure 8, we show the increased
effectiveness of Adjacency Filtering since less coordinates will be subject to
edit-distance calculation and Figure 9 shows the increased execution time. \\ 

For cheap key selection, we face the same problem.  Since we are picking the
cheapest N-e+1 keys. As e increases, the keys we pick will become less
\textit{cheap}. When e = N-1, the program will just behave as if without cheap
key selection, since it is picking all the keys anyhow. We can increase the
chance of picking relatively cheap keys by the same trick as above: smaller key
length and more segments. However, by also same with the reason listed above,
increasing the key number does not necessarily provides speed up. Although we
increase the chance picking relatively cheap keys, since the coordinate list is
longer now, the now \textit{cheap} keys’ coordinate list are actually longer
than before. We modeled key length 9 case, and the result is the shown in
Figure 7 as well. \\ 

In reality, normally error tolerance e is set to 5\% of the fragment sequence
length. For a 108 read, e will normally be 5, by which FashHASH still provides
generally considerable speed up. However, as e increases, the performance
improvement will diminish, because of the reason stated above. \\
