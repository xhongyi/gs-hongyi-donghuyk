\section{Algorithm Optimization} \label{sec:algorithm}

\subsection{Adjacency Filtering} \label{sec:af} 

mrFAST suffers from long execution time for several reasons. One of the biggest
reasons is that the edit-distance calculation is time consuming. The other
reason is that the hash-table is not balanced. By not balanced we mean some
patterns store more coordinates within its entry whereas some patterns store
fewer coordinates. For those patterns storing long lists of coordinates, we
call them expensive keys.  Expensive keys imply two problems: 1. Expensive keys
have long coordinate list, which means if ever used them as keys, they will
impose many string comparisons and hence many reference DNA database accesses.
2. In real test cases, they will be frequently encountered. This property is
associated to the first property—they are popular in human DNA so that they
have longer coordinate list. As a result, when sampling, we will have higher
chance encounter these expensive keys since they show up more frequently in
human DNA. \\
 
%%%%%%%%%%%%%%%%%%%%%%%%%%%%%%%%%%%%%%%%%%%%%%%%%%%%%%%%%%%%%%%%%%%%%%%%%%%%%%%%
\begin{figure}[t] 
\centering
\vspace{0.1in}
\includegraphics[width=3in]{./figure/Entry_Size_B.pdf} \vspace{0in}
\caption{The coordinate entry size}
\label{fig:entry_size} 
\end{figure}
%%%%%%%%%%%%%%%%%%%%%%%%%%%%%%%%%%%%%%%%%%%%%%%%%%%%%%%%%%%%%%%%%%%%%%%%%%%%%%%%

Figure~\ref{fig:entry_size} shows the coordinates entry size for chromosome 1.
For a vast majority of the entries, there is 0 coordinate stored inside which
means this pattern never shows up in this chromosome. However, there are also
some coordinates having more than thousands or even millions of coordinates. As
a result, whenever mrFAST uses those segments as keys, there will be thousands
to millions of edit-distance conducted.  Figure~\ref{fig:edit_dist} shows how many
edit-distance calculations performed turned out to be a match. When aligning 10
million fragments from chromosome 1 to chromosome 1, allowing max error number
to be 3 normally out of 3000 edit-distance calculations, 1 will be a match.
Such low matching rate implies that the majority of the coordinates subject to
edit-distance calculation will be rejected, at a high execution time cost.
From our analysis, it turns out some of the not matching coordinates can be
rejected solely by analyzing the information from the hash table alone, without
even accessing reference sequence database and performing the expensive
edit-distance calculation. Such early stage rejection would save memory
bandwidth, on chip cache as well as execution time.\\

%%%%%%%%%%%%%%%%%%%%%%%%%%%%%%%%%%%%%%%%%%%%%%%%%%%%%%%%%%%%%%%%%%%%%%%%%%%%%%%%
\begin{figure}[t] 
\centering
\vspace{0.1in}
\includegraphics[width=3in]{./figure/Edit_Dist_B.pdf} \vspace{0in}
\caption{The number of edit-distance performs}
\label{fig:edit_dist} 
\end{figure}
%%%%%%%%%%%%%%%%%%%%%%%%%%%%%%%%%%%%%%%%%%%%%%%%%%%%%%%%%%%%%%%%%%%%%%%%%%%%%%%%

The key observation behind this early rejection is that: if the fragment can be
divided into N segments, and we probe e+1 segments’ coordinate lists as we
described in 2.2.1. Then for each coordinate stored in the coordinate list, if
such coordinate is a potential match, the adjacent keys have adjacent locations
stored in their coordinate lists. An example will be the best explanation. For
instance, as shown in Figure~\ref{fig:ad_filtering}, a fragment f is divided into
N segments s1, s2… sN, with each segments being L in length. When probing the
coordinate entry for the first segment, on the first coordinate coor1 from s1,
we test if coor1 + L is located in the coordinate list of s2 and if coor1 + 2*L
is located in the coordinate list of s3, so on and so forth. For a perfect
match, all adjacent segments should locate at corresponding locations, which
means all corresponding locations should be stored inside the corresponding
coordinate list. However, for inexact matches, where we allow mismatches, the
constrain will be relaxed to at least N – e segments should find corresponding
locations in their coordinate list. The reasons why we relax the filtering
constrain from all segments to N – e segments is that now there could be at
most errors in the fragment, which in the worst case could be distributed in e
segments. In presents of insertion and deletions, the filtering test will be
further relaxed to the range of expected coordinate plus or minus e. For
example in Figure~\ref{fig:ad_filtering}, if now the error number e is set to 3,
then for coor1, instead of searching for coor1 + L in the coordinate list of
s2, we now search for coor1 + L +- e for s2 and coor1 + L*2 +-e for s3, so on
and so forth.  Additionally, instead of requiring matches for all of the
segments, the fragment will pass the filtering test if more than N – e segments
found corresponding locations. \\

%%%%%%%%%%%%%%%%%%%%%%%%%%%%%%%%%%%%%%%%%%%%%%%%%%%%%%%%%%%%%%%%%%%%%%%%%%%%%%%%
\begin{figure}[t] 
\centering
\vspace{0.1in}
\includegraphics[width=3in]{./figure/Adjacency_Filtering_B.pdf} \vspace{0in}
\caption{Adjacency Filtering}
\label{fig:ad_filtering} 
\end{figure}
%%%%%%%%%%%%%%%%%%%%%%%%%%%%%%%%%%%%%%%%%%%%%%%%%%%%%%%%%%%%%%%%%%%%%%%%%%%%%%%%

Adjacency Filtering itself will not guarantee matches but it will filter out
obvious not matching locations. If a fragment has more than N – e segments that
fail finding adjacent coordinates, we can deduce there must be more than e
errors thus rejecting the coordinate. However, if the fragment passes the
adjacency filtering, this does not necessarily mean the input fragment matches
to the reference DNA sequence within e errors. We cannot guarantee the total
error number is less than e since for those segments that failed finding
adjacent locations in their coordinate lists, they might have more than 1
errors in it.  Thus the adjacency filtering itself is not complete. For a
matching input fragment and reference DNA coordinate pair that passes adjacency
filtering process, we still have to perform edit distance calculation on them
to get the exact number and locations of the errors.  Although adjacency
filtering is not complete and cannot replace edit-distance calculation, it did
drastically reduced the times of edit-distance being called. We will show the
numbers in evaluation section.\\

\subsection{Cheap Key Selection} \label{sec:cheapkey}

\textbf{Pigeon Hole theorem and multiple search keys} There are several reasons
why we have to allow several errors for finding out potential matching
locations. First is the potential difference between individual DNA sequences.
Most portion of individual DNA sequence is similar with Reference DNA sequences
and using this similarity, we have tried to reconstruct individual DNA
sequences by matching it to the known Reference sequences. However, most of
important portions of DNA sequences are the “difference” between DNA sequences.
These differences can cause specific diseases or determine specific individual
characteristics. So, one of the most important thing for matching fragment to
reference sequence is to give the potential location within maximum allowable
errors, such as insertion, deletion and mismatch. Second reason is that there
are potential errors caused by DNA sequencing analyzer’s misreading. For
example, Illumina platform has relatively poor performance in assorting G-C,
i.e. Illumina platform frequently misread G as C or vice versa. The other NGS
platforms also have their own sequencing biases. By matching fragments to
Reference sequence with allowing specific number of errors, we can reduce the
matching failures caused by NGS platforms specific errors.  String comparison
operation of mrFAST can give the edit-distance between two sequences, which can
be the threshold values to decide whether the location can be the potential
location of fragment or not.  However, main assumption of hash table based
seed-and-extend algorithm is that there is no error in the fragment of key. If
we assume that the fragment has some errors caused by a certain reason, like
individual DNA difference or platform biases, and the errors are located at the
searching key for hash table, then, the coordinates according to the search key
have potential errors.  Finally, we lose the chance that we can find out exact
matching cases with allowing these errors.\\

The obvious solution of this potential problem is to select multiple search
keys within the fragment, which is based on Pigeon Hole theorem. Pigeon Hole
theorem is that if there are “m” pigeon and \textit{m+1} hole which can be
occupied by just one pigeon, then, one of the holes has to be blanked at least.
So, if \textit{m+1} keys are selected as search key to guarantee \textit{m}
allowable errors, then, one of the search keys don’t has errors at least.
Figure~\ref{fig:pigeon} shows the worst case of error distribution. 5 errors
distributed across the different keys. In this case, if we select any 6 keys of
total keys, we can get a key which do not has errors at least. 

%%%%%%%%%%%%%%%%%%%%%%%%%%%%%%%%%%%%%%%%%%%%%%%%%%%%%%%%%%%%%%%%%%%%%%%%%%%%%%%%
\begin{figure}[b] \centering \vspace{0.1in}
\includegraphics[width=2.5in]{./figure/pigeon_B.pdf} \vspace{0in}
\caption{Error distribution \& required number of keys}
\label{fig:pigeon} 
\end{figure}
%%%%%%%%%%%%%%%%%%%%%%%%%%%%%%%%%%%%%%%%%%%%%%%%%%%%%%%%%%%%%%%%%%%%%%%%%%%%%%%%

\textbf{Avoid duplicated Adjacency filtering at using multiple searching keys}
When we tried to use \textit{e+1} keys as searching keys of Adjacency
filtering, if the fragment is passed, then, it is possible that there are
duplicated Adjacency filtering computations. For example, if the fragment is
exactly matched to the Reference sequence at certain coordinate and we selected
\textit{e+1} keys as searching keys, then, each Adjacency filtering
corresponding to \textit{e+1} keys find out same coordinate as potentially
matching coordinate. This means that \textit{e+1} duplicated Adjacency
operations are computed for finding out just one possible location. We can
filter out these duplicated and redundant Adjacency operations by comparing the
expected coordinate as the result of Adjacency filtering and the coordinates,
already found as potentially matching coordinate before taking Adjacency
filtering.  We preserve the potentially matching coordinates, the result of
previous Adjacency filtering at specific sized, 100ea in our implementation,
array to filter out the redundant Adjacency filter.  If it is matched, then, we
simply cancel the reserved Adjacency filtering.

%%%%%%%%%%%%%%%%%%%%%%%%%%%%%%%%%%%%%%%%%%%%%%%%%%%%%%%%%%%%%%%%%%%%%%%%%%%%%%%%
\begin{figure}[t] \centering \vspace{0.1in}
\includegraphics[height=1.7in]{./figure/Key_Dist_B.pdf} \vspace{0in}
\caption{Adjacency Filtering Distribution} 
\label{fig:key_dist} 
\end{figure}
%%%%%%%%%%%%%%%%%%%%%%%%%%%%%%%%%%%%%%%%%%%%%%%%%%%%%%%%%%%%%%%%%%%%%%%%%%%%%%%%

\textbf{Imbalance of the number of key entry size} In section~\ref{sec:af},
Adjacency filtering drastically reduces the number of string comparison
perform. However, Adjacency filtering also requires large computational
resources for searching the expected coordinate in hash table to decide whether
this location can be potentially matched to fragment or not. The imbalance of
key entry size exacerbates the problem because we have to perform Adjacency
filtering as much as key entry size. Figure~\ref{fig:key_dist}  shows the
distribution of key entry size within one chromosome. If we use the fragments
which are located at \textit{A} position, about 150,000 Adjacency filtering performs
are required. The number of fragments located at “A” position is over 1000.
Now, the dominant computation requirement changes from the string comparison
perform to Adjacency Filtering perform.

%%%%%%%%%%%%%%%%%%%%%%%%%%%%%%%%%%%%%%%%%%%%%%%%%%%%%%%%%%%%%%%%%%%%%%%%%%%%%%%%
\begin{figure}[h] \centering \vspace{0.1in}
\includegraphics[width=3.0in]{./figure/Cheap_Key_B.pdf} \vspace{0in}
\caption{Cheap Key Selection Mechanism} 
\label{fig:cheap_key} 
\end{figure}
%%%%%%%%%%%%%%%%%%%%%%%%%%%%%%%%%%%%%%%%%%%%%%%%%%%%%%%%%%%%%%%%%%%%%%%%%%%%%%%%
%%%%%%%%%%%%%%%%%%%%%%%%%%%%%%%%%%%%%%%%%%%%%%%%%%%%%%%%%%%%%%%%%%%%%%%%%%%%%%%%
\begin{figure}[b] \centering \vspace{0.1in}
\includegraphics[width=3.0in]{./figure/CK_Result_B.pdf} \vspace{0in}
\caption{Cheap Key Selection Result} 
\label{fig:ck_result} 
\end{figure}
%%%%%%%%%%%%%%%%%%%%%%%%%%%%%%%%%%%%%%%%%%%%%%%%%%%%%%%%%%%%%%%%%%%%%%%%%%%%%%%%
%%%%%%%%%%%%%%%%%%%%%%%%%%%%%%%%%%%%%%%%%%%%%%%%%%%%%%%%%%%%%%%%%%%%%%%%%%%%%%%%
\begin{figure}[t] \centering \vspace{0.1in}
\includegraphics[height=1.7in]{./figure/Key_Dist2_B.pdf} \vspace{0in}
\caption{Efficiency of Cheap Key Selection} 
\label{fig:key_dist2} 
\end{figure}
%%%%%%%%%%%%%%%%%%%%%%%%%%%%%%%%%%%%%%%%%%%%%%%%%%%%%%%%%%%%%%%%%%%%%%%%%%%%%%%%

\textbf{Cheap Key Selection} If we allow \textit{e} errors, \textit{e+1}
searching keys have to be selected to maintain comprehensiveness. Each key has
its own key entry size. To avoid large Adjacent Filtering computation, we can
select the smallest \textit{e+1} keys as searching keys. Hash table already
stores the entry sizes corresponding to each key and based on this information,
we can easily sort the key at the order of entry size and select cheapest
\textit{e+1} keys as searching keys. In this case, we can drastically reduce
the number of Adjacency Filtering computations and increase the sequencing
performance without degrading comprehensiveness. We call this as Cheap Key
Selection. Figure~\ref{fig:cheap_key} shows the operation of Cheap Key
Selection.  First of all, we divide the fragments as keys, i.e. if each
fragment has 108ea base pairs and we use 12ea key length hash table, then we
can divide the fragments as 9 keys. After that, the keys of the fragment can be
sorted at the order of key entry size and cheapest \textit{e+1} keys can be
selected as searching keys. Figure~\ref{fig:key_dist2} shows the efficiency of
Cheap Key Selection. The blue colored diagram is the original distribution of
Adjacency Filtering and the red colored diagram is the distribution after Cheap
Key Selection. The number of the fragments, which requires large Adjacency
Filtering perform is drastically reduced.  Figure~\ref{fig:ck_result} shows
that the total number of Adjacency Filtering perform is reduced to 6.4\% of
original total number of Adjacency Filtering at the case of selecting first
\textit{e+1} keys as searching keys without sorting based on the entry size. \\
The Cheap Key Selection also requires computational resources and the number of
computations for sorting is proportional to log (the number of keys within each
fragment).  However, the number of Adjacency filtering is proportional to (the
number of keys within fragment) * log (the key entry size).  Usually, the
fragments have 100~200 base pairs and the key length of Hash Table is about 10
to 12, So, the cost of Cheap Key Selection computation is quite smaller than
the cost of Adjacency filtering. \\

